
\documentclass[]{article}

% Imported Packages
%------------------------------------------------------------------------------
\usepackage{amssymb}
\usepackage{amstext}
\usepackage{amsthm}
\usepackage{amsmath}
\usepackage{enumerate}
\usepackage{fancyhdr}
\usepackage[margin=1in]{geometry}
\usepackage{graphicx}
%\usepackage{extarrows}
\usepackage{setspace}
%------------------------------------------------------------------------------

% Header and Footer
%------------------------------------------------------------------------------
\pagestyle{plain}  
\renewcommand\headrulewidth{0.4pt}                                      
\renewcommand\footrulewidth{0.4pt}                                    
%------------------------------------------------------------------------------

% Title Details
%------------------------------------------------------------------------------
         
\begin{document}

\begin{titlepage}

\title{Software Requirements Specification}
\author{Kevin Hardy-Cooper 1312836\\
		Nareshkumar Maheshkumar 1320375\\
		Athidya Raveenthranehru 1316204\\
		Radhika Rani Sharma	1150430\\
		Mario Calce 1304792\\
		Abishek Mukherjee 1151803 }
\date{2016/02/08}

\maketitle

\end{titlepage}
\tableofcontents                  \newpage    
%------------------------------------------------------------------------------

% Document
%------------------------------------------------------------------------------
	

\section{Introduction}
\label{sec:introduction}
% Begin Section


 The following document will outline and describe the mobile application meant to answer the question ``What is this?'' with respect to natural optical phenomena.  In this document, the question of what the application will do will be addressed, however emphasis will be made to avoid addressing how the mobile application will accomplish its tasks. This document will define the purpose and scope of the application, along with any associated definitions, acronyms and abbreviations. In addition, product perspective, product function, user characteristics, constraints, and assumptions and dependencies will be explored with respect to how they apply to the mobile application. Functional requirements and non functional requirements will also be defined in this document.

\subsection{Purpose}
\label{sub:purpose}
% Begin SubSection

 As mentioned previously, the requirements document is meant to describe everything that the system must do. This will include all tasks that the application must complete as shown in the functional requirements, as well as how the application must look, perform, be maintained, be usable and be secure as shown in the non functional requirements. Through the use of business events and viewpoints, the requirements document will outline how the application must react to different user stimulus. The intended audience of this document is the teaching assistants as well as the professor for the course Software Engineering - Large System Design. 
% End SubSection

\subsection{Scope}
\label{sub:scope}
% Begin SubSection

The application will be called NatureOptix.  The product will allow the user to answer a set of questions asked by the application. The application will then try to determine what natural phenomena the user is trying to specify based off the users answers. The application will also allow the user to take pictures of the phenomena and post it to social media. The objective of this application will be to enable the user to identify natural and optic phenomena. Other features, such as allowing the user to post picture to social media will help to generate awareness of different natural and optic phenomena. 

% End SubSection

\subsection{Definitions, Acronyms, and Abbreviations}
\label{sub:definitions_acronyms_and_abbreviations}
% Begin SubSection

Not Applicable

% End SubSection

\subsection{References}
\label{sub:references}
% Begin SubSection
Not Applicable
% End SubSection

\subsection{Overview}
\label{sub:overview}
% Begin SubSection
\indent The rest of this document will be organized into 3 parts as follows: Overall Description, Functional Requirements, and Non-Functional Requirements. Each of these sections are further broken down. The Overall Description will discuss product perspective, product function, user characteristics, constraints, assumptions and dependencies, and apportioning of requirements. The Functional Requirement will list all functional requirements and provide business events and viewpoint corresponding to each. Non-Functional Requirements will be divided into the following sections: Look and Feel requirements, Usability and Humanity Requirements, Performance Requirements, Operational and Environmental Requirements, Cultural and Political requirements, and Legal Requirements. 

% End SubSection

% End Section

\section{Overall Description}
\label{sec:overall_description}
% Begin Section

%This section of the SRS should describe the general factors that affect the product and its requirements. It does not state specific requirements; it provides a background for those requirements and makes them easier to understand.

\subsection{Product Perspective}
\label{sub:product_perspective}
% Begin SubSection
%\begin{enumerate}[a)]
%	\item Put the product into perspective with other related products, i.e., context
%	\item If the product is independent and totally self-contained, it should be stated here
%	\item If the SRS defines a product that is a component of a larger system, as frequently occurs, then this subsection should relate the requirements of that larger system to functionality of the software and should identify interfaces between that system and the software
%	\item A block diagram showing the major components of the larger system, interconnections, and external interfaces can be helpful
%\end{enumerate}
% End SubSection
The product under production will be similar to Akinator, the Web Genie. Akinator is an Internet game that is based off of twenty questions. The game has the user think of a character or famous person. Akinator then asks the user a series of questions that can be answered with either Yes, No, Maybe, or Don't Know. Akinator than guess who the user is thinking of based on the answer given.  Unlike Akinator, this application will focus solely on identifying natural phenomenon based on user information.\\
Due to the time constraints of the semester the application will be totally self-contained. 
\subsection{Product Functions}
\label{sub:product_functions}
% Begin SubSection
%\begin{enumerate}[a)]
%	\item Provide a summary of the major functions that the software will perform.
%	\begin{itemize}
%		\item \textbf{Example}: An SRS for an accounting program may use this part to address customer account maintenance, customer statement, and invoice preparation without mentioning the vast amount of detail that each of those functions requires.
%	\end{itemize}
%	\item Functions should be organized in a way that makes the list of functions understandable to the customer or to anyone else reading the document for the first time
%	\item Textual or graphical methods can be used to show the different functions and their relationships
%	\begin{itemize}
%		\item Such a diagram is not intended to show a design of a product, but simply shows the logical relationships among variables
%	\end{itemize} 
%\end{enumerate}
% End SubSection
Some of the functions that the application will perform are as follows;
\begin{itemize}
\item Identify a natural phenomena based on inputs
\item Allow the user to take a photo to share on social media
\end{itemize}

\subsection{User Characteristics}
%need the requirements to complete
\label{sub:user_characteristics}
% Begin SubSection
The intended users are assumed to have the following characteristics:
\begin{itemize}
	%\item Describe those general characteristics of the intended users of the product including educational level, experience, and technical expertise
	\item User has a grade nine level education
	\item Users are aged at least thirteen years
	\item Users are familiar with using mobile applications
	\item Users have accounts with social media websites 
	  
	%\item Do not state specific requirements, but rather provide the reasons why certain specific requirements are later specified
\end{itemize}
These assumptions are made of the user because these are the characteristics of the indented audience of the application. 
% End SubSection

\subsection{Constraints}

\label{sub:constraints}
% Begin SubSection
Some of the constraints that have been put on the project are;
\begin{itemize}
	%\item Provide a general description of any other items that will limit the developer's options
	\item Time: The project must be completed within the set time of the semester. Also the developers time will be split between the project and other course work for external classes. 
	\item Budget: The project has a budget of zero dollars because it is a school project. 
	\item Software: The application is restricted to run on the Android operating system. This is a constraint that is laid out by the instructor for the project. 
\end{itemize}
% End SubSection

\subsection{Assumptions and Dependencies}
% need to see the actual requirements to complete.
\label{sub:assumptions_and_dependencies}
% Begin SubSection
The following assumptions are being made;
\begin{itemize}
	\item Assuming that users have phones that run at least Android 4.0.  
	\item Assuming that the user's phone has Internet access. 
	\item Assuming that the user's phone has a build in camera. 
	%\item 
\end{itemize}
The first assumption is to allow for the widest availability coverage. The reason for this is not all user will be running the same version of operating system. Building for the earliest allows for a wider user base. The second assumption is to ensure that the requirement for the use of Google Maps is met as well as allowing for the special feature of sharing the phenomenon on social media. The third assumption allows for the special feature of letting users photograph the phenomenon. 

% End SubSection

\subsection{Apportioning of Requirements}
\label{sub:apportioning_of_requirements}
% Begin SubSection
%	\item Identify requirements that may be delayed until future versions of the system
Not Applicable
% End SubSection 
% End Section

\section{Functional Requirements}
\label{sec:functional_requirements}
% Begin Section


\begin{enumerate}[{BE}1.]
	\item User wishes to access the application
	\begin{enumerate}[{VP1}.1]
		\item User
			\begin{enumerate}
				\item User shall be able to download the application onto their smart phone.  This operation will be handled by the operating system.
				\item User shall be able to open the application.
			\end{enumerate}
		\item System Developer
			\begin{enumerate}
				\item Application shall be able to handle user input.
			\end{enumerate}
	\end{enumerate}
	\item User wishes to take a picture and post it to social media
	\begin{enumerate}[{VP2}.1]
		\item User
			\begin{enumerate}
				\item User shall be able to access the built-in camera on phone and its functionality from the application.
				\item User shall be able to save the picture to the application and the phone.
				\item User shall be able to post to social media directly from the application.
			\end{enumerate}
		\item System Developer
			\begin{enumerate}
				\item Application shall have access to Internet via wireless connection from smart phone.
				\item Application shall have access to the built-in camera on the smart phone.
				\item Application shall have access to social media (Instagram)
				\item Application shall be able to save pictures directly to the phone.
				\item Application shall allow users to post pictures of observed phenomenon.
			\end{enumerate}
	\end{enumerate}
	\item User wishes to view and modify pictures from the application
	\begin{enumerate} [{VP3}.1]
		\item User
			\begin{enumerate}
				\item User shall be able to view saved pictures through the application and the phone.
				\item User shall be able delete pictures from the application and the phone.
			\end{enumerate}
		\item System Developer
			\begin{enumerate}
				\item Application shall display requested pictures to the user.
				\item Application shall be able to delete pictures directly on the phone.
			\end{enumerate}
	\end{enumerate}
	\item User wishes to identify a natural phenomena
	\begin{enumerate}[{VP4}.1]
		\item User
			\begin{enumerate}
				\item User shall be presented with options in which they can narrow down possible answers to "What is this?".
				\item User shall be able to post a picture on social media through the application to get feedback from friends to identify a phenomenon.
								
			\end{enumerate}
		\item System Developer
			\begin{enumerate}
				\item Application shall provide the user with multiple-choice questions to identify a natural phenomenon.
				\item Application shall access the user's location using Google maps services to assist with identifying phenomenon.
				\item Application shall display the identified phenomenon on the interface.
				\item Application shall have access to the Internet via wireless connection from smart phone.
				 
				
				
			\end{enumerate}
	\end{enumerate}


	\item Software Developer wishes to swap expert
	\begin{enumerate}[{VP1}.5.1]
		\item User
			\begin{enumerate}
				\item Not Applicable (Note: User should not see implementation)
			\end{enumerate}
		\item System Developer
			\begin{enumerate}
			\item Application shall be able to switch "expert" modules in order to identify the natural phenomenon.
			\item Application shall allow the software developer to swap experts with other experts.
			\end{enumerate}
	\end{enumerate}
\end{enumerate}
% End Section

\section{Non-Functional Requirements}
\label{sec:non-functional_requirements}
% Begin Section
\subsection{Look and Feel Requirements}
\label{sub:look_and_feel_requirements}
% Begin SubSection

\subsubsection{Appearance Requirements}
\label{ssub:appearance_requirements}
% Begin SubSubSection
Not Applicable
% End SubSubSection

\subsubsection{Style Requirements}
\label{ssub:style_requirements}
% Begin SubSubSection
%\begin{enumerate}[{LF}2. ]
	 Not Applicable
%\end{enumerate}
% End SubSubSection

% End SubSection

\subsection{Usability and Humanity Requirements}
\label{sub:usability_and_humanity_requirements}
% Begin SubSection

\subsubsection{Ease of Use Requirements}
\label{ssub:ease_of_use_requirements}
% Begin SubSubSection
\begin{itemize}
	\item The application shall be intuitively navigate-able for the intended users with minimal support. 
\end{itemize}
% End SubSubSection

\subsubsection{Personalization and Internationalization Requirements}
\label{ssub:personalization_and_internationalization_requirements}
% Begin SubSubSection
\begin{itemize}
	\item The application shall be available in Canadian English.
	\item The application shall allow the user to use their personal Instagram account to upload images. 
\end{itemize}
% End SubSubSection

\subsubsection{Learning Requirements}
\label{ssub:learning_requirements}
% Begin SubSubSection
Not Applicable
% End SubSubSection

\subsubsection{Understandability and Politeness Requirements}
\label{ssub:understandability_and_politeness_requirements}
% Begin SubSubSection
\begin{itemize}
	\item The application shall use symbols and words that are understandable to the intended user.
	\item The application shall not offend the user in any way.
\end{itemize}
% End SubSubSection

\subsubsection{Accessibility Requirements}
\label{ssub:accessibility_requirements}
% Begin SubSubSection
%\begin{enumerate}[{UH}1. ]
	Not Applicable
%\end{enumerate}
% End SubSubSection

% End SubSection

\subsection{Performance Requirements}
\label{sub:performance_requirements}
% Begin SubSection

\subsubsection{Speed and Latency Requirements}
\label{ssub:speed_and_latency_requirements}
% Begin SubSubSection
\begin{itemize}
	\item The application shall respond to any user input within three seconds.
	\item The application shall allow the user to upload an image to their Instagram account within two minutes of initialization.
	\item The application shall receive the user's location within two minutes of initialization.
\end{itemize}
% End SubSubSection

\subsubsection{Safety-Critical Requirements}
\label{ssub:safety_critical_requirements}
% Begin SubSubSection
%\begin{enumerate}[{PR}4. ]
Not Applicable
%\end{enumerate}
% End SubSubSection

\subsubsection{Precision or Accuracy Requirements}
\label{ssub:precision_or_accuracy_requirements}
% Begin SubSubSection
\begin{itemize}
	\item The application shall detect location according to the accuracy available within the Google maps API.
\end{itemize}
% End SubSubSection

\subsubsection{Reliability and Availability Requirements}
\label{ssub:reliability_and_availability_requirements}
% Begin SubSubSection
\begin{itemize}
	\item The application shall be available for use 24 hours a day, 364 days a year. One day will be reserved for updates and maintenance. 
	\item The application shall be available on the Google Play Store for download.
\end{itemize}
	

% End SubSubSection

\subsubsection{Robustness or Fault-Tolerance Requirements}
\label{ssub:robustness_or_fault_tolerance_requirements}
% Begin SubSubSection
\begin{itemize}
	\item The application shall alert user if Internet connection is not available.
\end{itemize}
% End SubSubSection

\subsubsection{Capacity Requirements}
\label{ssub:capacity_requirements}
% Begin SubSubSection
\begin{itemize}
	\item The application shall be able to save as many photos as there is memory available on the device.
\end{itemize}
% End SubSubSection

\subsubsection{Scalability or Extensibility Requirements}
\label{ssub:scalability_or_extensibility_requirements}
% Begin SubSubSection

Not Applicable

% End SubSubSection

\subsubsection{Longevity Requirements}
\label{ssub:longevity_requirements}
% Begin SubSubSection
\begin{itemize}
	\item The application shall be able to operate with minimal maintenance.
\end{itemize}
% End SubSubSection

% End SubSection

\subsection{Operational and Environmental Requirements}
\label{sub:operational_and_environmental_requirements}
% Begin SubSection

\subsubsection{Expected Physical Environment}
\label{ssub:expected_physical_environment}
% Begin SubSubSection
\begin{itemize}
	\item The application shall be able to operate on any android mobile device.
\end{itemize}

% End SubSubSection

\subsubsection{Requirements for Interfacing with Adjacent Systems}
\label{ssub:requirements_for_interfacing_with_adjacent_systems}
% Begin SubSubSection
\begin{itemize}
	\item The application shall interact with Google maps in order to determine the user's location.
	\item The application shall be able to capture photos using the mobile device's camera application.
\end{itemize}
% End SubSubSection

\subsubsection{Productization Requirements}
\label{ssub:productization_requirements}
% Begin SubSubSection
%\begin{enumerate}[{OE}1. ]
Not Applicable
%\end{enumerate}
% End SubSubSection

\subsubsection{Release Requirements}
\label{ssub:release_requirements}
% Begin SubSubSection
%\begin{enumerate}[{OE}1. ]
	Not Applicable
%\end{enumerate}
% End SubSubSection

% End SubSection


\subsection{Maintainability and Support Requirements}
\label{sub:maintainability_and_support_requirements}
% Begin SubSection

\subsubsection{Maintenance Requirements}
\label{ssub:maintenance_requirements}
% Begin SubSubSection
\begin{itemize}
		
	    \item The application shall require maintenance one day per year. 
        	    
\end{itemize}
% End SubSubSection

\subsubsection{Supportability Requirements}
\label{ssub:supportability_requirements}
% Begin SubSubSection
\begin{itemize}
	\item The application shall have a help menu for users.
\end{itemize}
% End SubSubSection

\subsubsection{Adaptability Requirements}
\label{ssub:adaptability_requirements}
% Begin SubSubSection
Not Applicable
% End SubSubSection

% End SubSection

\subsection{Security Requirements}
\label{sub:security_requirements}
% Begin SubSection

\subsubsection{Access Requirements}
\label{ssub:access_requirements}
% Begin SubSubSection
\begin{itemize}
	\item The application shall only access the user's mobile device camera with permission. 
	\item The application shall only access the user's Instagram account with permission. 
\end{itemize}
% End SubSubSection

\subsubsection{Integrity Requirements}
\label{ssub:integrity_requirements}
% Begin SubSubSection
\begin{itemize}
	\item The application shall not access folders on the user's mobile device that it does not have permission to.
\end{itemize}
% End SubSubSection

\subsubsection{Privacy Requirements}
\label{ssub:privacy_requirements}
% Begin SubSubSection
\begin{itemize}
	\item The application shall not share the user's location without permission. 
\end{itemize}
% End SubSubSection

\subsubsection{Audit Requirements}
\label{ssub:audit_requirements}
% Begin SubSubSection
\begin{itemize}
	\item The application shall share system data with the appropriate authorities upon audit. 
\end{itemize}
% End SubSubSection

\subsubsection{Immunity Requirements}
\label{ssub:immunity_requirements}
% Begin SubSubSection
\begin{itemize}
	\item The application shall not store the user's private information.
	\end{itemize}
% End SubSubSection

% End SubSection

\subsection{Cultural and Political Requirements}
\label{sub:cultural_and_political_requirements}
% Begin SubSection

\subsubsection{Cultural Requirements}
\label{ssub:cultural_requirements}
% Begin SubSubSection
\begin{itemize}
	\item The application shall not offend any cultures.
\end{itemize}
% End SubSubSection

\subsubsection{Political Requirements}
\label{ssub:political_requirements}
% Begin SubSubSection
Not Applicable
% End SubSubSection

% End SubSection

\subsection{Legal Requirements}
\label{sub:legal_requirements}
% Begin SubSection

\begin{itemize}
	\item The application shall not break any existing laws.
\end{itemize}
%End SubSection
\subsubsection{Compliance Requirements}
\label{ssub:compliance_requirements}
% Begin SubSubSection
Not Applicable
% End SubSubSection

\subsubsection{Standards Requirements}
\label{ssub:standards_requirements}
% Begin SubSubSection
\begin{itemize}
	\item The application shall formally ask permission from the user when accessing other mobile device applications.
\end{itemize}
% End SubSubSection

% End SubSection

% End Section

\appendix
\section{Division of Labour}
\label{sec:division_of_labour}
 %Begin Section
Kevin John Hardy-Cooper - Functional Requirements
\newline Nareshkumar Maheshkumar - Functional Requirements
\newline Athidya Raveenthranehru - Non-Functional Requirements
\newline Radhika Rani Sharma - Introduction
\newline Mario Calce - Overall Description
\newline Abhishek Mukherjee - Non-Functional Requirements
% End Section

%\newpage
%\section*{IMPORTANT NOTES}
%\begin{itemize}
%	\item Be sure to include all sections of the template in your document regardless whether you have something to write for each or not
%	\begin{itemize}
%		\item If you do not have anything to write in a section, indicate this by the \emph{N/A}, \emph{void}, \emph{none}, etc.
%	\end{itemize}
%	\item Uniquely number each of your requirements for easy identification and cross-referencing
%	\item Highlight terms that are defined in Section~1.3 (\textbf{Definitions, Acronyms, and Abbreviations}) with \textbf{bold}, \ emph{italic} or \underline{underline}
%	\item For Deliverable 1, please highlight, in some fashion, all (you may have more than one) creative and innovative features. Your creative and innovative features will generally be described in Section~2.2 (\textbf{Product Functions}), but it will depend on the type of creative or innovative features you are including.
%\end{itemize}


\end{document}
%------------------------------------------------------------------------------
