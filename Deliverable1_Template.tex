\documentclass[]{article}

% Imported Packages
%------------------------------------------------------------------------------
\usepackage{amssymb}
\usepackage{amstext}
\usepackage{amsthm}
\usepackage{amsmath}
\usepackage{enumerate}
\usepackage{fancyhdr}
\usepackage[margin=1in]{geometry}
\usepackage{graphicx}
\usepackage{extarrows}
\usepackage{setspace}
%------------------------------------------------------------------------------

% Header and Footer
%------------------------------------------------------------------------------
\pagestyle{plain}  
\renewcommand\headrulewidth{0.4pt}                                      
\renewcommand\footrulewidth{0.4pt}                                    
%------------------------------------------------------------------------------

% Title Details
%------------------------------------------------------------------------------
\title{Deliverable \#1}
\author{SE 3A04: Software Design II -- Large System Design}
\date{2016/02/03}                               
%------------------------------------------------------------------------------

% Document
%------------------------------------------------------------------------------
\begin{document}

\maketitle	

\section{Introduction}
\label{sec:introduction}
% Begin Section


The following document will outline and describe the mobile application meant to answer the question ``What is this?'' with respect to natural optical phenomena.  In this document, the question of what the application will do will be addressed, however emphasis will be made to avoid addressing how the mobile application will accomplish its tasks. This document will define the purpose and scope of the application, along with any associated definitions, acronyms and abbreviations. In addition, product perspective, product function, user characteristics, constraints, and assumptions and dependencies will be explored with respect to how they apply to the mobile application. Functional requirements and non functional requirements will also be defined in this document.

\subsection{Purpose}
\label{sub:purpose}
% Begin SubSection
\begin{enumerate}[a)]
	\item As mentioned previously, the requirements document is meant to describe everything that the system must do. This will include all tasks that the application must complete as shown in the functional requirements , as well as how the application must look, perform, be maintained, be usable and be secure as shown in the non functional requirements. Through the use of business events and viewpoints, the requirements document will outline how the application must react to different user stimulus.
	\item The intended audience of this document is anyone who wishes to gain insight on what the application must do and how it must react to user stimulus.
\end{enumerate}
% End SubSection

\subsection{Scope}
\label{sub:scope}
% Begin SubSection
\begin{enumerate}[a)]
	\item The application will be called Nature Optix.
	\item The product will allow the user to answer a set of questions asked by the application. The application will then try to determine what natural phenomena the user is trying to specify based off the users answers. The application will also allow the user to take pictures of the phenomena and post it to social media. 
	\item The objective of this application will be to enable the user to identify natural optic phenomena. Other features, such as allowing the user to post picture to social media will help to generate awareness of different natural optic phenomena. 
\end{enumerate}
% End SubSection

\subsection{Definitions, Acronyms, and Abbreviations}
\label{sub:definitions_acronyms_and_abbreviations}
% Begin SubSection
\begin{enumerate}[a)]
	\item REVISIT THIS
\end{enumerate}
% End SubSection

\subsection{References}
\label{sub:references}
% Begin SubSection
Not Applicable
% End SubSection

\subsection{Overview}
\label{sub:overview}
% Begin SubSection
\begin{enumerate}[a)]
	\item The rest of this document will be organized into 3 parts as follows: Overall Description, Functional Requirements, and Non-Functional Requirements. Each of these sections are further broken down. The Overall Description will discuss product perspective, product function, user characteristics, constraints, assumptions and dependencies, and apportioning of requirements. The Functional Requirement will list all functional requirements and provide business events and viewpoint corresponding to each. Non-Functional Requirements will be divided into the following sections: Look and Feel requirements, Usability and Humanity Requirements, Performance Requirements, Operational and Environmental Requirements, Cultural and Political requirements, and Legal Requirements. 
\end{enumerate}
% End SubSection

% End Section

\section{Overall Description}
\label{sec:overall_description}
% Begin Section

This section of the SRS should describe the general factors that affect the product and its requirements. It does not state specific requirements; it provides a background for those requirements and makes them easier to understand.

\subsection{Product Perspective}
\label{sub:product_perspective}
% Begin SubSection
\begin{enumerate}[a)]
	\item Put the product into perspective with other related products, i.e., context
	\item If the product is independent and totally self-contained, it should be stated here
	\item If the SRS defines a product that is a component of a larger system, as frequently occurs, then this subsection should relate the requirements of that larger system to functionality of the software and should identify interfaces between that system and the software
	\item A block diagram showing the major components of the larger system, interconnections, and external interfaces can be helpful
\end{enumerate}
% End SubSection

\subsection{Product Functions}
\label{sub:product_functions}
% Begin SubSection
\begin{enumerate}[a)]
	\item Provide a summary of the major functions that the software will perform.
	\begin{itemize}
		\item \textbf{Example}: An SRS for an accounting program may use this part to address customer account maintenance, customer statement, and invoice preparation without mentioning the vast amount of detail that each of those functions requires.
	\end{itemize}
	\item Functions should be organized in a way that makes the list of functions understandable to the customer or to anyone else reading the document for the first time
	\item Textual or graphical methods can be used to show the different functions and their relationships
	\begin{itemize}
		\item Such a diagram is not intended to show a design of a product, but simply shows the logical relationships among variables
	\end{itemize} 
\end{enumerate}
% End SubSection

\subsection{User Characteristics}
\label{sub:user_characteristics}
% Begin SubSection
\begin{enumerate}[a)]
	\item Describe those general characteristics of the intended users of the product including educational level, experience, and technical expertise
	\item Do not state specific requirements, but rather provide the reasons why certain specific requirements are later specified
\end{enumerate}
% End SubSection

\subsection{Constraints}
\label{sub:constraints}
% Begin SubSection
\begin{enumerate}[a)]
	\item Provide a general description of any other items that will limit the developer's options
\end{enumerate}
% End SubSection

\subsection{Assumptions and Dependencies}
\label{sub:assumptions_and_dependencies}
% Begin SubSection
\begin{enumerate}[a)]
	\item List each of the factors that affect the requirements stated in the SRS
	\item These factors are not design constraints on the software but are, rather, any changes to them that can affect the requirements in the SRS
	\begin{itemize}
		\item \textbf{Example}: An assumption may be that a specific operating system will be available on the hardware designated for the software product. If, in fact, the operating system is not available, the SRS would then have to change accordingly.
	\end{itemize}
\end{enumerate}
% End SubSection

\subsection{Apportioning of Requirements}
\label{sub:apportioning_of_requirements}
% Begin SubSection
\begin{enumerate}[a)]
	\item Identify requirements that may be delayed until future versions of the system
\end{enumerate}
% End SubSection

% End Section

\section{Functional Requirements}
\label{sec:functional_requirements}
% Begin Section
This section of the SRS should contain all of the software requirements to a level of detail sufficient to enable designers to design a system to satisfy those requirements, and testers to test that the system satisfies those requirements. Throughout this section, every stated requirement should be externally perceivable by users, operators, or other external systems. These requirements should include at a minimum a description of every input (stimulus) into the system, every output (response) from the system, and all functions performed by the system in response to an input or in support of an output.

\begin{enumerate}[{BE}1.]
	\item User wishes to access the app
	\begin{enumerate}[{VP1}.1]
		\item User
			\begin{enumerate}
				\item User shall be able to download the app onto their smart phone.
				\item User shall be able to open the app.
			\end{enumerate}
		\item App
			\begin{enumerate}
				\item App shall be able to handle user input.
			\end{enumerate}
	\end{enumerate}
	\item User wishes to take a picture and post it to social media
	\begin{enumerate}[{VP2}.1]
		\item User
			\begin{enumerate}
				\item User shall be able to access the built-in camera on phone from the app.
				\item User shall be able to take a picture.
				\item User shall be able to save the picture to the app and the phone.
				\item User shall be able to post to social media directly from the app.
				\item User shall be able to view saved pictures through the app.
				\item User shall be able delete pictures from the app and the phone.
			\end{enumerate}
		\item App
			\begin{enumerate}
				\item App shall have access to internet via wireless connection from smart phone.
				\item App shall have access to the built-in camera on the smart phone.
				\item App shall have access to social media (Instagram)
				\item App shall display requested pictures to the user.
				\item App shall be able to save pictures directly to the phone.
				\item App shall be able to delete pictures directly on the phone.
			\end{enumerate}
	\end{enumerate}
	\item User wishes to identify a natural phenomena
	\begin{enumerate}[{VP2}.1]
		\item User
			\begin{enumerate}
				\item User shall be presented with options in which they can narrow down possible answers to "What is this?".
				\item User shall be able to post a picture on social media through the app to get feedback from friends to identify a phenomenon.
								
			\end{enumerate}
		\item App
			\begin{enumerate}
				\item App shall provide the user "Yes or No" questions to identify a natural phenomenon.
				\item App shall access the user's location using google maps services.
				\item App shall be able to narrow down options after each question asked. 
				\item App shall display the identified phenomenon on the interface.
				\item App shall have access to the internet via wireless connection from smart phone.
				\item App shall have access to social media(Instagram). 
				
				
			\end{enumerate}
	\end{enumerate}
\end{enumerate}

% End Section

\section{Non-Functional Requirements}
\label{sec:non-functional_requirements}
% Begin Section
\subsection{Look and Feel Requirements}
\label{sub:look_and_feel_requirements}
% Begin SubSection

\subsubsection{Appearance Requirements}
\label{ssub:appearance_requirements}
% Begin SubSubSection
\begin{enumerate}[{LF}1. ]
	\item The product shall have a simple display and easy to understand user interface.
\end{enumerate}
% End SubSubSection

\subsubsection{Style Requirements}
\label{ssub:style_requirements}
% Begin SubSubSection
\begin{enumerate}[{LF}1. ]
	\item N/A
\end{enumerate}
% End SubSubSection

% End SubSection

\subsection{Usability and Humanity Requirements}
\label{sub:usability_and_humanity_requirements}
% Begin SubSection

\subsubsection{Ease of Use Requirements}
\label{ssub:ease_of_use_requirements}
% Begin SubSubSection
\begin{enumerate}[{UH}1. ]
	\item Product should be understandable and easily navigateable for all those between the ages seven to sixty years old.
	\item No prior training is necessary in order to use this product assuming the user understands the basic naviagtion of their phone and general applications on their device.
\end{enumerate}
% End SubSubSection

\subsubsection{Personalization and Internationalization Requirements}
\label{ssub:personalization_and_internationalization_requirements}
% Begin SubSubSection
\begin{enumerate}[{UH}1. ]
	\item Product is only available in Canadian English.
	\item Each user will be able to use their personal social media accounts to upload their desired images. The social media platforms that they will be able to access are Facebook, Instagram and Twitter. %any other social media?
\end{enumerate}
% End SubSubSection

\subsubsection{Learning Requirements}
\label{ssub:learning_requirements}
% Begin SubSubSection
\begin{enumerate}[{UH}1. ]
	\item Agreed percentage of a test panel shall successfully complete [specified task] within [specified time limit]. %clarify 
	\item Assumption is made that user already knows the basics of navigating their phone and simple applications on their phone.
\end{enumerate}
% End SubSubSection

\subsubsection{Understandability and Politeness Requirements}
\label{ssub:understandability_and_politeness_requirements}
% Begin SubSubSection
\begin{enumerate}[{UH}1. ]
	\item The product shall hide the details of its implementation from the user.
	\item The product shall use symbols and words that are naturally understandable to the user.
\end{enumerate}
% End SubSubSection

\subsubsection{Accessibility Requirements}
\label{ssub:accessibility_requirements}
% Begin SubSubSection
\begin{enumerate}[{UH}1. ]
	\item N/A
\end{enumerate}
% End SubSubSection

% End SubSection

\subsection{Performance Requirements}
\label{sub:performance_requirements}
% Begin SubSection

\subsubsection{Speed and Latency Requirements}
\label{ssub:speed_and_latency_requirements}
% Begin SubSubSection
\begin{enumerate}[{PR}1. ]
	\item The system shall respond to any user input within three seconds.
	\item The user shall be able to upload the desired picture to their designated social media platform within two minutes.
	\item The user shall be able to recieve their location status within two minutes. %clarify specs before final draft
\end{enumerate}
% End SubSubSection

\subsubsection{Safety-Critical Requirements}
\label{ssub:safety_critical_requirements}
% Begin SubSubSection
\begin{enumerate}[{PR}1. ]
	\item N/A
\end{enumerate}
% End SubSubSection

\subsubsection{Precision or Accuracy Requirements}
\label{ssub:precision_or_accuracy_requirements}
% Begin SubSubSection
\begin{enumerate}[{PR}1. ]
	\item The product shall accurately detect natural phenomenon 80\% of the time. %should be 100?
	\item Location shall be accurately dected according to google maps accuracy levels %research
\end{enumerate}
% End SubSubSection

\subsubsection{Reliability and Availability Requirements}
\label{ssub:reliability_and_availability_requirements}
% Begin SubSubSection
\begin{enumerate}[{PR}1. ]
	\item Product shall be available for use 24 hours a day, every day of the year.
\end{enumerate}
% End SubSubSection

\subsubsection{Robustness or Fault-Tolerance Requirements}
\label{ssub:robustness_or_fault_tolerance_requirements}
% Begin SubSubSection
\begin{enumerate}[{PR}1. ]
	\item The product shall alert user if internet connection is not available. %confirm if true
\end{enumerate}
% End SubSubSection

\subsubsection{Capacity Requirements}
\label{ssub:capacity_requirements}
% Begin SubSubSection
\begin{enumerate}[{PR}1. ]
	\item The product shall be able to save as many photos as there is memory available on the device. %confirm
\end{enumerate}
% End SubSubSection

\subsubsection{Scalability or Extensibility Requirements}
\label{ssub:scalability_or_extensibility_requirements}
% Begin SubSubSection
\begin{enumerate}[{PR}1. ]
	\item N/A
\end{enumerate}
% End SubSubSection

\subsubsection{Longevity Requirements}
\label{ssub:longevity_requirements}
% Begin SubSubSection
\begin{enumerate}[{PR}1. ]
	\item This product is expected to operate without any maintenance.
\end{enumerate}
% End SubSubSection

% End SubSection

\subsection{Operational and Environmental Requirements}
\label{sub:operational_and_environmental_requirements}
% Begin SubSection

\subsubsection{Expected Physical Environment}
\label{ssub:expected_physical_environment}
% Begin SubSubSection
\begin{enumerate}[{OE}1. ]
	\item N/A
\end{enumerate}
% End SubSubSection

\subsubsection{Requirements for Interfacing with Adjacent Systems}
\label{ssub:requirements_for_interfacing_with_adjacent_systems}
% Begin SubSubSection
\begin{enumerate}[{OE}1. ]
	\item This product shall interact with google maps in order to determine the user's location and location of natural phenomenon.
	\item This product shall be able to capture photos using the devices camera application.
	\item This product shall be able to upload photos to user's social media platforms including Facebook, Twitter and Instagram.
	\item This product shall be able to interact with weather application to determine current weather status. %other experts too
\end{enumerate}
% End SubSubSection

\subsubsection{Productization Requirements}
\label{ssub:productization_requirements}
% Begin SubSubSection
\begin{enumerate}[{OE}1. ]
	\item N/A %unless there is a specific method we are using for downloads
\end{enumerate}
% End SubSubSection

\subsubsection{Release Requirements}
\label{ssub:release_requirements}
% Begin SubSubSection
\begin{enumerate}[{OE}1. ]
	\item N/A %or no future plans for product as of yet
\end{enumerate}
% End SubSubSection

% End SubSection

\subsection{Maintainability and Support Requirements}
\label{sub:maintainability_and_support_requirements}
% Begin SubSection

\subsubsection{Maintenance Requirements}
\label{ssub:maintenance_requirements}
% Begin SubSubSection
\begin{enumerate}[{MS}1. ]
	\item 
\end{enumerate}
% End SubSubSection

\subsubsection{Supportability Requirements}
\label{ssub:supportability_requirements}
% Begin SubSubSection
\begin{enumerate}[{MS}1. ]
	\item 
\end{enumerate}
% End SubSubSection

\subsubsection{Adaptability Requirements}
\label{ssub:adaptability_requirements}
% Begin SubSubSection
\begin{enumerate}[{MS}1. ]
	\item 
\end{enumerate}
% End SubSubSection

% End SubSection

\subsection{Security Requirements}
\label{sub:security_requirements}
% Begin SubSection

\subsubsection{Access Requirements}
\label{ssub:access_requirements}
% Begin SubSubSection
\begin{enumerate}[{SR}1. ]
	\item 
\end{enumerate}
% End SubSubSection

\subsubsection{Integrity Requirements}
\label{ssub:integrity_requirements}
% Begin SubSubSection
\begin{enumerate}[{SR}1. ]
	\item 
\end{enumerate}
% End SubSubSection

\subsubsection{Privacy Requirements}
\label{ssub:privacy_requirements}
% Begin SubSubSection
\begin{enumerate}[{SR}1. ]
	\item 
\end{enumerate}
% End SubSubSection

\subsubsection{Audit Requirements}
\label{ssub:audit_requirements}
% Begin SubSubSection
\begin{enumerate}[{SR}1. ]
	\item 
\end{enumerate}
% End SubSubSection

\subsubsection{Immunity Requirements}
\label{ssub:immunity_requirements}
% Begin SubSubSection
\begin{enumerate}[{SR}1. ]
	\item 
\end{enumerate}
% End SubSubSection

% End SubSection

\subsection{Cultural and Political Requirements}
\label{sub:cultural_and_political_requirements}
% Begin SubSection

\subsubsection{Cultural Requirements}
\label{ssub:cultural_requirements}
% Begin SubSubSection
\begin{enumerate}[{CP}1. ]
	\item 
\end{enumerate}
% End SubSubSection

\subsubsection{Political Requirements}
\label{ssub:political_requirements}
% Begin SubSubSection
\begin{enumerate}[{CP}1. ]
	\item 
\end{enumerate}
% End SubSubSection

% End SubSection

\subsection{Legal Requirements}
\label{sub:legal_requirements}
% Begin SubSection

\subsubsection{Compliance Requirements}
\label{ssub:compliance_requirements}
% Begin SubSubSection
\begin{enumerate}[{LR}1. ]
	\item 
\end{enumerate}
% End SubSubSection

\subsubsection{Standards Requirements}
\label{ssub:standards_requirements}
% Begin SubSubSection
\begin{enumerate}[{LR}1. ]
	\item 
\end{enumerate}
% End SubSubSection

% End SubSection

% End Section

\appendix
\section{Division of Labour}
\label{sec:division_of_labour}
% Begin Section
Include a Division of Labour sheet which indicates the contributions of each team member. This sheet must be signed by all team members.
% End Section

\newpage
\section*{IMPORTANT NOTES}
\begin{itemize}
	\item Be sure to include all sections of the template in your document regardless whether you have something to write for each or not
	\begin{itemize}
		\item If you do not have anything to write in a section, indicate this by the \emph{N/A}, \emph{void}, \emph{none}, etc.
	\end{itemize}
	\item Uniquely number each of your requirements for easy identification and cross-referencing
	\item Highlight terms that are defined in Section~1.3 (\textbf{Definitions, Acronyms, and Abbreviations}) with \textbf{bold}, \emph{italic} or \underline{underline}
	\item For Deliverable 1, please highlight, in some fashion, all (you may have more than one) creative and innovative features. Your creative and innovative features will generally be described in Section~2.2 (\textbf{Product Functions}), but it will depend on the type of creative or innovative features you are including.
\end{itemize}


\end{document}
%------------------------------------------------------------------------------
