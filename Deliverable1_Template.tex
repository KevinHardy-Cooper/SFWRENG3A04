
\documentclass[]{article}

% Imported Packages
%------------------------------------------------------------------------------
\usepackage{amssymb}
\usepackage{amstext}
\usepackage{amsthm}
\usepackage{amsmath}
\usepackage{enumerate}
\usepackage{fancyhdr}
\usepackage[margin=1in]{geometry}
\usepackage{graphicx}
%\usepackage{extarrows}
\usepackage{setspace}
%------------------------------------------------------------------------------

% Header and Footer
%------------------------------------------------------------------------------
\pagestyle{plain}  
\renewcommand\headrulewidth{0.4pt}                                      
\renewcommand\footrulewidth{0.4pt}                                    
%------------------------------------------------------------------------------

% Title Details
%------------------------------------------------------------------------------
         
\begin{document}

\begin{titlepage}

\title{Software Requirements Specification}
\author{Kevin Hardy-Cooper 1312836\\
		Nareshkumar Maheshkumar 1320375\\
		Athidya Raveenthranehru 1316204\\
		Radhika Rani Sharma	1150430\\
		Mario Calce 1304792\\
		Abishek Mukherjee 1151803 }
\date{2016/02/08}

\maketitle

\end{titlepage}
\tableofcontents                  \newpage    
%------------------------------------------------------------------------------

% Document
%------------------------------------------------------------------------------
	

\section{Introduction}
\label{sec:introduction}
% Begin Section


 The following document will outline and describe the mobile application meant to answer the question ``What is this?'' with respect to natural optical phenomena.  In this document, the question of what the application will do will be addressed, however emphasis will be made to avoid addressing how the mobile application will accomplish its tasks. This document will define the purpose and scope of the application, along with any associated definitions, acronyms and abbreviations. In addition, product perspective, product function, user characteristics, constraints, and assumptions and dependencies will be explored with respect to how they apply to the mobile application. Functional requirements and non functional requirements will also be defined in this document.

\subsection{Purpose}
\label{sub:purpose}
% Begin SubSection

 As mentioned previously, the requirements document is meant to describe everything that the system must do. This will include all tasks that the application must complete as shown in the functional requirements, as well as how the application must look, perform, be maintained, be usable and be secure as shown in the non functional requirements. Through the use of business events and viewpoints, the requirements document will outline how the application must react to different user stimulus. The intended audience of this document is the teaching assistants as well as the proffesor. 
% End SubSection

\subsection{Scope}
\label{sub:scope}
% Begin SubSection

The application will be called Nature Optix.  The product will allow the user to answer a set of questions asked by the application. The application will then try to determine what natural phenomena the user is trying to specify based off the users answers. The application will also allow the user to take pictures of the phenomena and post it to social media. The objective of this application will be to enable the user to identify natural and optic phenomena. Other features, such as allowing the user to post picture to social media will help to generate awareness of different natural and optic phenomena. 

% End SubSection

\subsection{Definitions, Acronyms, and Abbreviations}
\label{sub:definitions_acronyms_and_abbreviations}
% Begin SubSection

Not Applicable

% End SubSection

\subsection{References}
\label{sub:references}
% Begin SubSection
Not Applicable
% End SubSection

\subsection{Overview}
\label{sub:overview}
% Begin SubSection
\indent The rest of this document will be organized into 3 parts as follows: Overall Description, Functional Requirements, and Non-Functional Requirements. Each of these sections are further broken down. The Overall Description will discuss product perspective, product function, user characteristics, constraints, assumptions and dependencies, and apportioning of requirements. The Functional Requirement will list all functional requirements and provide business events and viewpoint corresponding to each. Non-Functional Requirements will be divided into the following sections: Look and Feel requirements, Usability and Humanity Requirements, Performance Requirements, Operational and Environmental Requirements, Cultural and Political requirements, and Legal Requirements. 

% End SubSection

% End Section

\section{Overall Description}
\label{sec:overall_description}
% Begin Section

%This section of the SRS should describe the general factors that affect the product and its requirements. It does not state specific requirements; it provides a background for those requirements and makes them easier to understand.

\subsection{Product Perspective}
\label{sub:product_perspective}
% Begin SubSection
%\begin{enumerate}[a)]
%	\item Put the product into perspective with other related products, i.e., context
%	\item If the product is independent and totally self-contained, it should be stated here
%	\item If the SRS defines a product that is a component of a larger system, as frequently occurs, then this subsection should relate the requirements of that larger system to functionality of the software and should identify interfaces between that system and the software
%	\item A block diagram showing the major components of the larger system, interconnections, and external interfaces can be helpful
%\end{enumerate}
% End SubSection
The product under production will be similar to Akinator, the Web Genie. Akinator is an internet game that is based off of twenty questions. The game has the user think of a charcter or famous person. Akinator then asks the user a series of questions that can be answered with either Yes, No, Maybe, or Don't Know. Akinator than guess who the user is thinking of based on the answer given.  Unlike Akinator, this application will focus soley on identifying natural phenomenon based on user information.\\
Due to the time constraints of the semester the application will be totally self-contained. 
\subsection{Product Functions}
\label{sub:product_functions}
% Begin SubSection
%\begin{enumerate}[a)]
%	\item Provide a summary of the major functions that the software will perform.
%	\begin{itemize}
%		\item \textbf{Example}: An SRS for an accounting program may use this part to address customer account maintenance, customer statement, and invoice preparation without mentioning the vast amount of detail that each of those functions requires.
%	\end{itemize}
%	\item Functions should be organized in a way that makes the list of functions understandable to the customer or to anyone else reading the document for the first time
%	\item Textual or graphical methods can be used to show the different functions and their relationships
%	\begin{itemize}
%		\item Such a diagram is not intended to show a design of a product, but simply shows the logical relationships among variables
%	\end{itemize} 
%\end{enumerate}
% End SubSection
Some of the funtions that the application will perform are as follows;
\begin{itemize}
\item Identify a natural phenomona based on inputs
\item Allow the user to take a photo to share on social media
\end{itemize}

\subsection{User Characteristics}
%need the requirements to complete
\label{sub:user_characteristics}
% Begin SubSection
The intended users are assumed to have the following characteristics:
\begin{itemize}
	%\item Describe those general characteristics of the intended users of the product including educational level, experience, and technical expertise
	\item User has a grade nine level education
	\item Users are aged at least thirteen years
	\item Users are familiar with using mobile applications
	\item Users have accounts with social media websites 
	  
	%\item Do not state specific requirements, but rather provide the reasons why certain specific requirements are later specified
\end{itemize}
These assumptions are made of the user because these are the characterics of the indented audience of the application. 
% End SubSection

\subsection{Constraints}

\label{sub:constraints}
% Begin SubSection
Some of the constraints that have been put on the project are;
\begin{itemize}
	%\item Provide a general description of any other items that will limit the developer's options
	\item Time: The project must be completed within the set time of the sesmester. Also the developers time will be spilt between the project and other course work for external classes. 
	\item Budget: The project has a budget of zero dollars becuase it is a school project. 
	\item Software: The application is restricted to run on the Android operating system. This is a constraint that is laid out by the insturcutor for the project. 
\end{itemize}
% End SubSection

\subsection{Assumptions and Dependencies}
% need to see the actual requirements to complete.
\label{sub:assumptions_and_dependencies}
% Begin SubSection
The following assumptions are being made;
\begin{itemize}
	\item Assuming that users have phones that run atleast Android 4.0.  
	\item Assuming that the user's phone has internet access. 
	\item Assuming that the user's phone has a build in camera. 
	%\item 
\end{itemize}
The first assumption is to allow for the widest availability coverage. The reason for this is not all user will be running the same version of operating system. Building for the earliest allows for a wider user base. The second assumption is to ensure that the requirement for the use of Google Maps is met as well as allowing for the speacial feature of sharing the phenomenon on social media. The third assumption allows for the speacial feature of letting users photograph the pheomenon. 

% End SubSection

\subsection{Apportioning of Requirements}
\label{sub:apportioning_of_requirements}
% Begin SubSection
%	\item Identify requirements that may be delayed until future versions of the system
Not Applicable
% End SubSection 
% End Section

\section{Functional Requirements}
\label{sec:functional_requirements}
% Begin Section


\begin{enumerate}[{BE}1.]
	\item User wishes to access the application
	\begin{enumerate}[{VP1}.1]
		\item User
			\begin{enumerate}
				\item User shall be able to download the application onto their smart phone.  This operation will be handled by the operating system.
				\item User shall be able to open the application.
			\end{enumerate}
		\item System Developer
			\begin{enumerate}
				\item Application shall be able to handle user input.
			\end{enumerate}
	\end{enumerate}
	\item User wishes to take a picture and post it to social media
	\begin{enumerate}[{VP2}.1]
		\item User
			\begin{enumerate}
				\item User shall be able to access the built-in camera on phone and its functionality from the application.
				\item User shall be able to save the picture to the application and the phone.
				\item User shall be able to post to social media directly from the application.
			\end{enumerate}
		\item System Developer
			\begin{enumerate}
				\item Application shall have access to internet via wireless connection from smart phone.
				\item Application shall have access to the built-in camera on the smart phone.
				\item Application shall have access to social media (Instagram)
				\item Application shall be able to save pictures directly to the phone.
			\end{enumerate}
	\end{enumerate}
	\item User wishes to view and modify pictures from the application
	\begin{enumerate} [{VP3}.1]
		\item User
			\begin{enumerate}
				\item User shall be able to view saved pictures through the application and the phone.
				\item User shall be able delete pictures from the application and the phone.
			\end{enumerate}
		\item System Developer
			\begin{enumerate}
				\item Application shall display requested pictures to the user.
				\item Application shall be able to delete pictures directly on the phone.
			\end{enumerate}
	\end{enumerate}
	\item User wishes to identify a natural phenomena
	\begin{enumerate}[{VP4}.1]
		\item User
			\begin{enumerate}
				\item User shall be presented with options in which they can narrow down possible answers to "What is this?".
				\item User shall be able to post a picture on social media through the app to get feedback from friends to identify a phenomenon.
								
			\end{enumerate}
		\item System Developer
			\begin{enumerate}
				\item Application shall provide the user "Yes or No" questions to identify a natural phenomenon.
				\item Application shall access the user's location using google maps services to assist with identifying phenomenon.
				\item Application shall be able to narrow down options after each question asked. 
				\item Application shall display the identified phenomenon on the interface.
				\item Application shall have access to the internet via wireless connection from smart phone.
				\item Application shall have access to social media(Instagram)to allow users to post pictures of observed phenomenon. 
				\item Application shall be able to switch "expert" modules in order to identify the natural phenomenon.
				
			\end{enumerate}
	\end{enumerate}
\end{enumerate}

% End Section

\section{Non-Functional Requirements}
\label{sec:non-functional_requirements}
% Begin Section
\subsection{Look and Feel Requirements}
\label{sub:look_and_feel_requirements}
% Begin SubSection

\subsubsection{Appearance Requirements}
\label{ssub:appearance_requirements}
% Begin SubSubSection
\begin{itemize}
	N/A
\end{itemize}
% End SubSubSection

\subsubsection{Style Requirements}
\label{ssub:style_requirements}
% Begin SubSubSection
%\begin{enumerate}[{LF}2. ]
	 Not Applicable
%\end{enumerate}
% End SubSubSection

% End SubSection

\subsection{Usability and Humanity Requirements}
\label{sub:usability_and_humanity_requirements}
% Begin SubSection

\subsubsection{Ease of Use Requirements}
\label{ssub:ease_of_use_requirements}
% Begin SubSubSection
\begin{itemize}
	\item Product should be understandable and easily navigateable for all those between the ages seven to sixty years old.
	\item No prior training should necessary in order to use this product assuming the user understands the basic naviagtion of their phone and general applications on their device.
\end{itemize}
% End SubSubSection

\subsubsection{Personalization and Internationalization Requirements}
\label{ssub:personalization_and_internationalization_requirements}
% Begin SubSubSection
\begin{itemize}
	\item Applicaton is written in Canadian English.
	\item Each user will be able to use their personal social media accounts to upload their desired images. The social media platforms that they will be able to access are Facebook, Instagram and Twitter.
\end{itemize}
% End SubSubSection

\subsubsection{Learning Requirements}
\label{ssub:learning_requirements}
% Begin SubSubSection
\begin{itemize}
	N/A
\end{itemize}
% End SubSubSection

\subsubsection{Understandability and Politeness Requirements}
\label{ssub:understandability_and_politeness_requirements}
% Begin SubSubSection
\begin{itemize}
	\item The product shall hide the details of its implementation from the user.
	\item The product shall use symbols and words that are naturally understandable to the user.
\end{itemize}
% End SubSubSection

\subsubsection{Accessibility Requirements}
\label{ssub:accessibility_requirements}
% Begin SubSubSection
%\begin{enumerate}[{UH}1. ]
	Not Applicable
%\end{enumerate}
% End SubSubSection

% End SubSection

\subsection{Performance Requirements}
\label{sub:performance_requirements}
% Begin SubSection

\subsubsection{Speed and Latency Requirements}
\label{ssub:speed_and_latency_requirements}
% Begin SubSubSection
\begin{itemize}
	\item The system shall respond to any user input within three seconds.
	\item The user shall be able to upload the desired picture to their designated social media platform within two minutes.
	\item The user shall be able to recieve their location status within two minutes.
\end{itemize}
% End SubSubSection

\subsubsection{Safety-Critical Requirements}
\label{ssub:safety_critical_requirements}
% Begin SubSubSection
%\begin{enumerate}[{PR}4. ]
Not Applicable
%\end{enumerate}
% End SubSubSection

\subsubsection{Precision or Accuracy Requirements}
\label{ssub:precision_or_accuracy_requirements}
% Begin SubSubSection
\begin{itemize}
	\item Location shall be accurately detected according to accuracy available with google maps.
\end{itemize}
% End SubSubSection

\subsubsection{Reliability and Availability Requirements}
\label{ssub:reliability_and_availability_requirements}
% Begin SubSubSection
\begin{itemize
	\item Product shall be available for use 24 hours a day, every day of the year except for when the application is undergoing an update.
\end{itemize}
% End SubSubSection

\subsubsection{Robustness or Fault-Tolerance Requirements}
\label{ssub:robustness_or_fault_tolerance_requirements}
% Begin SubSubSection
\begin{itemize}
	\item The product shall alert user if internet connection is not available.
\end{itemize}
% End SubSubSection

\subsubsection{Capacity Requirements}
\label{ssub:capacity_requirements}
% Begin SubSubSection
\begin{itemize}
	\item The product shall be able to save as many photos as there is memory available on the device.
\end{itemize}
% End SubSubSection

\subsubsection{Scalability or Extensibility Requirements}
\label{ssub:scalability_or_extensibility_requirements}
% Begin SubSubSection
\begin{itemize}
\item When the application scales, depending on our algorithm and the web services that we end up using, we need to take into consideration that each user is served equally and well (request reponse less than 5 sec).
\end{itemize}
% End SubSubSection

\subsubsection{Longevity Requirements}
\label{ssub:longevity_requirements}
% Begin SubSubSection
\begin{itemize}
	\item This product is expected to operate without any maintenance.
\end{itemize}
% End SubSubSection

% End SubSection

\subsection{Operational and Environmental Requirements}
\label{sub:operational_and_environmental_requirements}
% Begin SubSection

\subsubsection{Expected Physical Environment}
\label{ssub:expected_physical_environment}
% Begin SubSubSection
Not Applicable

% End SubSubSection

\subsubsection{Requirements for Interfacing with Adjacent Systems}
\label{ssub:requirements_for_interfacing_with_adjacent_systems}
% Begin SubSubSection
\begin{itemize}
	\item This product shall interact with google maps in order to determine the user's location and location of natural phenomenon.
	\item This product shall be able to capture photos using the devices camera application.
	\item This product shall be able to upload photos to user's social media platforms including Facebook, Twitter and Instagram.
	\item This product shall be able to interact with weather application to determine current weather status.
\end{itemize}
% End SubSubSection

\subsubsection{Productization Requirements}
\label{ssub:productization_requirements}
% Begin SubSubSection
%\begin{enumerate}[{OE}1. ]
	
	Not Applicable
%\end{enumerate}
% End SubSubSection

\subsubsection{Release Requirements}
\label{ssub:release_requirements}
% Begin SubSubSection
%\begin{enumerate}[{OE}1. ]
	Not Applicable
%\end{enumerate}
% End SubSubSection

% End SubSection


\subsection{Maintainability and Support Requirements}
\label{sub:maintainability_and_support_requirements}
% Begin SubSection

\subsubsection{Maintenance Requirements}
\label{ssub:maintenance_requirements}
% Begin SubSubSection
\begin{itemize}
	\item The application needs to be able to work efficently under several circumstances, namely - scalability, portability and robustness.
		\begin{itemize}
	    \item Portability : For now the application should only work for Android. But we need to make sure that the architecture is so robust that come time and if we get a good response, we can port it to other platforms
        \end{itemize}	    
\end{itemize}
% End SubSubSection

\subsubsection{Supportability Requirements}
\label{ssub:supportability_requirements}
% Begin SubSubSection
\begin{itemize}
	\item We shall take multiple steps to ensure that the application has built in/dedicated support via :-
	    \begin{itemize}
	    \item Secondary/Backup Server
	    \item Framework/Frameworks that support backing up of data
	    \item Dedicated support personnel
	    \item A forum for users
	    \end{itemize}
\end{itemize}
% End SubSubSection

\subsubsection{Adaptability Requirements}
\label{ssub:adaptability_requirements}
% Begin SubSubSection
\begin{itemize}
	\item Depending on how well the app does on the Android platform, we may decide to push it out on iOS, BlackBerry and Web.
\end{itemize}
% End SubSubSection

% End SubSection

\subsection{Security Requirements}
\label{sub:security_requirements}
% Begin SubSection

\subsubsection{Access Requirements}
\label{ssub:access_requirements}
% Begin SubSubSection
\begin{itemize}
	\item Anyone with an Android phone (version 4.4 and up) will be able to download the application via the Play Store and use the it.
\end{itemize}
% End SubSubSection

\subsubsection{Integrity Requirements}
\label{ssub:integrity_requirements}
% Begin SubSubSection
\begin{itemize}
	\item The application shall prevent users from getting an incorrect response to a request. The algorithm (experts) will ensure that the response is accurate.
\end{itemize}
% End SubSubSection

\subsubsection{Privacy Requirements}
\label{ssub:privacy_requirements}
% Begin SubSubSection
\begin{itemize}
	\item The product shall not be obtaining crucial private information from the user, however in terms of being open about our intentions, we will be notifying users everytime we change our information policy.
\end{itemize}
% End SubSubSection

\subsubsection{Audit Requirements}
\label{ssub:audit_requirements}
% Begin SubSubSection
\begin{itemize}
	\item All application data; user and otherwise, shall be turned over to the authorities upon an audit. This information will be mentioned in the Information Policy.
\end{itemize}
% End SubSubSection

\subsubsection{Immunity Requirements}
\label{ssub:immunity_requirements}
% Begin SubSubSection
\begin{itemize}
	\item The application is being developed for the Android platform; a derivative of linux, which is good in terms of handling malicious hacking attempts. We will be implementing Symmetric security measures for the different facets of the application. Emphasis will be on securing the database to ensure correct information is obtained upon every user request.
\end{itemize}
% End SubSubSection

% End SubSection

\subsection{Cultural and Political Requirements}
\label{sub:cultural_and_political_requirements}
% Begin SubSection

\subsubsection{Cultural Requirements}
\label{ssub:cultural_requirements}
% Begin SubSubSection
\begin{itemize}
	\item The application should make sure that none of the phenomena (especially phenomena that is specific to a certian country/part of the world) has any cultural/religious affliations/statements associated to it.
\end{itemize}
% End SubSubSection

\subsubsection{Political Requirements}
\label{ssub:political_requirements}
% Begin SubSubSection
\begin{itemize}
	\item The logo of the application shall be decided upon by the group leader. All feature decisions of the application shall be discussed and implemented upon by the developers.
\end{itemize}
% End SubSubSection

% End SubSection

\subsection{Legal Requirements}
\label{sub:legal_requirements}
% Begin SubSection

\subsubsection{Compliance Requirements}
\label{ssub:compliance_requirements}
% Begin SubSubSection
\begin{itemize}
	\item Personal information shall be implemented so as to comply with the
	Data Protection Act.
\end{itemize}
% End SubSubSection

\subsubsection{Standards Requirements}
\label{ssub:standards_requirements}
% Begin SubSubSection
\begin{itemize}
	\item The applications shall comply with the standards stated by all the experts (web services) that are being used.
\end{itemize}
% End SubSubSection

% End SubSection

% End Section

\appendix
\section{Division of Labour}
\label{sec:division_of_labour}
 %Begin Section
Kevin John Hardy-Cooper - Functional Requirements\\
Nareshkumar Maheshkumar - Functional Requirements\\
Athidya Raveenthranehru - Non-Functional Requirements\\
Radhika Rani Sharma - Introduction\\
Mario Calce - Overall Description\\
Abhishek Mukherjee - Non-Functional Requirements\\
% End Section

%\newpage
%\section*{IMPORTANT NOTES}
%\begin{itemize}
%	\item Be sure to include all sections of the template in your document regardless whether you have something to write for each or not
%	\begin{itemize}
%		\item If you do not have anything to write in a section, indicate this by the \emph{N/A}, \emph{void}, \emph{none}, etc.
%	\end{itemize}
%	\item Uniquely number each of your requirements for easy identification and cross-referencing
%	\item Highlight terms that are defined in Section~1.3 (\textbf{Definitions, Acronyms, and Abbreviations}) with \textbf{bold}, \ emph{italic} or \underline{underline}
%	\item For Deliverable 1, please highlight, in some fashion, all (you may have more than one) creative and innovative features. Your creative and innovative features will generally be described in Section~2.2 (\textbf{Product Functions}), but it will depend on the type of creative or innovative features you are including.
%\end{itemize}


\end{document}
%------------------------------------------------------------------------------
